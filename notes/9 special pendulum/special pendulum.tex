\documentclass[10pt]{article}

\usepackage{amssymb,amsmath,amsthm}
\usepackage{bm}
\usepackage{graphicx,subcaption}
\usepackage[letterpaper, top=1in, left=1in, right=1in, bottom=1in]{geometry}

\newtheorem{definition}{Definition}
\newtheorem{theorem}{Theorem}
\newtheorem{lemma}{Lemma}
\newtheorem{remark}{Remark}

\newcommand{\SO}{\ensuremath{\mathrm{SO}(3)}}
\newcommand{\tr}[1]{\ensuremath{\mathrm{tr}\left( #1 \right)}}
\newcommand{\abs}[1]{\ensuremath{\left| #1 \right|}}
\newcommand{\diff}[1]{\mathrm{d}#1}
\newcommand{\vect}[1]{\ensuremath{\mathrm{vec}\left[ #1 \right]}}

\newcommand{\liediff}{\mathfrak{d}}
\newcommand{\dft}{\mathcal{F}}
\newcommand{\real}{\ensuremath{\mathbb{R}}}
\newcommand{\sph}{\ensuremath{\mathbb{S}}}
\newcommand{\diag}{\ensuremath{\mathrm{diag}}}

\begin{document}

\section{Equations of Motion}

The general equations of motion for a 3D pendulum are given as
\begin{align}
	R^T\diff{R} &= \hat{\Omega}\diff{t}, \\
	J\diff{\Omega} &= \left( -\Omega\times J\Omega - mg\rho\times R^Te_3 \right) \diff{t} + JH\diff{W}_t.
\end{align}
We want to simplify the equations so that the pendulum does not rotate about its body-fixed axial axis.
Suppose that $J = \diag(J_1,J_2,J_3)$ is diagonal, and $J_1 = J_2$, then $(\Omega\times J\Omega)_3 = J_2\Omega_1\Omega_2 - J_1\Omega_1\Omega_2 = 0$.
Further assume that $\rho = \begin{bmatrix} 0 & 0 & \rho_z \end{bmatrix}$, then $(\rho\times R^Te_3)_3 = 0$.
So $d\Omega_3 = 0$ as long as that the third row of $H$ is also zero.
Let $\Omega_3(t_0) = 0$, $\Omega_3(t)=0$ for all $t\geq t_0$, so the pendulum does not rotate about its $e_3$-axis.
With these assumptions, the equations of motion can be simplified as
\begin{align}
	R^T\diff{R} &= \left( \begin{bmatrix} \Omega_1 & \Omega_2 & 0 \end{bmatrix}^T \right)^\wedge \diff{t}, \\
	\diff{\begin{bmatrix} \Omega_1 \\ \Omega_2 \end{bmatrix}} &= \frac{mg\rho_z}{J_1} \begin{bmatrix} R_{32} \\ -R_{31} \end{bmatrix} \diff{t} + \begin{bmatrix} H_{11} & H_{12} & H_{13} \\ H_{21} & H_{22} & H_{23} \end{bmatrix} \diff{W_t}.
\end{align}

\section{Collision Response}

Suppose the pendulum is cylindrical with radius $r$ and height $h$

\end{document}

