\documentclass[10pt]{article}

\usepackage{amssymb,amsmath,amsthm}
\usepackage{bm}
\usepackage{graphicx,subcaption}
\usepackage[letterpaper, top=1in, left=1in, right=1in, bottom=1in]{geometry}

\newtheorem{definition}{Definition}
\newtheorem{theorem}{Theorem}
\newtheorem{lemma}{Lemma}
\newtheorem{remark}{Remark}

\newcommand{\SO}{\ensuremath{\mathrm{SO}(3)}}
\newcommand{\tr}[1]{\ensuremath{\mathrm{tr}\left( #1 \right)}}
\newcommand{\abs}[1]{\ensuremath{\left| #1 \right|}}
\newcommand{\diff}[1]{\mathrm{d}#1}
\newcommand{\vect}[1]{\ensuremath{\mathrm{vec}\left[ #1 \right]}}

\newcommand{\liediff}{\mathfrak{d}}
\newcommand{\dft}{\mathcal{F}}
\newcommand{\real}{\ensuremath{\mathbb{R}}}
\newcommand{\sph}{\ensuremath{\mathbb{S}}}
\newcommand{\diag}{\ensuremath{\mathrm{diag}}}

\title{\vspace{-4ex}\textbf{Uncertainty Propagation for a 3D Pendulum\vspace{-4ex}}}
\date{}

\begin{document}

\maketitle

The dynamical model for a hinged 3D pendulum is
\begin{align*}
	R^T\diff{R} &= \hat{\Omega}\diff{t} \\
	J\diff{\Omega} &= \left( -\Omega\times J\Omega - mg\rho\times R^Te_3 \right) \diff{t} + H\diff{W}_t,
\end{align*}
where $J\in\real^{3\times 3}$ is the moment of inertia matrix, $\rho$ is the vector from the pivot point to the center of mass expressed in the body-fixed frame.
The Fokker-Planck equation is
\begin{align*}
	\frac{\partial p(R,\Omega,t)}{\partial t} = -\sum_{i=1}^{3} \liediff_i (\Omega_ip(R,\Omega,t)) + \sum_{i=1}^{3} \frac{\partial}{\partial \Omega_i} \left[(\Omega\times J\Omega + mg\rho\times R^Te_3)_i p(R,\Omega,t)\right] + \sum_{i,j=1}^{3} G_{ij} \frac{\partial^2 p(R,\Omega,t)}{\partial \Omega_i \partial \Omega_j}.
\end{align*}
Let $F^l_{m,n,i,j,k}[p]$ be the Fourier coefficient of the probability density, then it satisfies the following ordinary different equation
\begin{align} \label{eqn:FP Fourier}
	\frac{\diff{F^l_{m,n,i,j,k}[p]}}{\diff{t}} = &-\sum_{\alpha=1}^3 F^l_{m,n,i,j,k}[\liediff_\alpha (\Omega_\alpha p)] + \sum_{\alpha=1}^3 F^l_{m,n,i,j,k}\left[ \frac{\partial}{\partial \Omega_\alpha}[(\Omega\times J\Omega + mg\rho\times R^Te_3)_\alpha p] \right] \nonumber \\
	&+ \sum_{\alpha,\beta=1}^3 G_{\alpha\beta} F^l_{m,n,i,j,k}\left[ \frac{\partial^2 p}{\partial\Omega_\alpha \partial\Omega_\beta} \right].
\end{align}

\section{Marginal Density}

\begin{lemma} \label{lemma:marginal Fourier}
	Let $f(R,x)\in\mathcal{L}^2(\SO\times\real^N)$, and $F^l_{m,n,\mathcal{J}}$ be its Fourier coefficients, where $\mathcal{J} = \{j_1,\ldots,j_N\} \in \mathcal{N}^N$.
	Also, let $\tilde{f}(x) = \int_{R\in\SO} f(R,x) \diff{R}$, and $\tilde{F}_\mathcal{J}$ be its Fourier coefficients.
	Then
	\begin{align} \label{eqn:F_J}
		\tilde{F}_{\mathcal{J}} = F^0_{0,0,\mathcal{J}}.
	\end{align}
\end{lemma}
\begin{proof}
	First, $f(R,x)$ can be expanded as
	\begin{align*}
		f(R,x) = \sum_{l=0}^\infty \sum_{m,n=-l}^l \sum_{j'_1,\ldots,j'_N=-\infty}^\infty (2l+1) F^l_{m,n,\mathcal{J}'} D^l_{m,n}(R) \exp\left( \frac{2\pi i}{L} \sum_{\alpha=1}^N j'_\alpha x_\alpha \right).
	\end{align*}
	So we have
	\begin{align} \label{eqn:F_J calculation}
		\tilde{F}_\mathcal{J} &= \frac{1}{L^N} \int_{x\in[-L/2,L/2]^N} \int_{R\in\SO} f(R,x) \diff{R} \exp\left( -\frac{2\pi i}{L} \sum_{\alpha=1}^N j_\alpha x_\alpha \right) \diff{x} \nonumber \\
		&= \frac{1}{L^N} \sum_{j'_1,\ldots,j'_N=-\infty}^\infty \bigg[ \int_{x\in[-L/2,L/2]^N}  \exp\left( \frac{2\pi i}{L} \sum_{\alpha=1}^N j'_\alpha x_\alpha \right) \exp\left( -\frac{2\pi i}{L} \sum_{\alpha=1}^N j_\alpha x_\alpha \right) \diff{x} \nonumber \\
		&\qquad\qquad \times \int_{R\in\SO} \sum_{l=0}^{\infty} \sum_{m,n=-l}^l (2l+1)F^l_{m,n,\mathcal{J}'} D^l_{m,n}(R) \diff{R} \bigg] \nonumber \\
		&= \sum_{j'_1,\ldots,j'_N=-\infty}^\infty \delta_\mathcal{J}^{\mathcal{J}'} \sum_{l=0}^\infty \sum_{m,n=-l}^l (2l+1)F^l_{m,n,\mathcal{J}'} \int_{R\in\SO} D^l_{m,n}(R) \diff{R}.
	\end{align}
	Next, the integral of $D^l_{m,n}(R)$ can be directly calculated as
	\begin{align*}
		\int_{R\in\SO} D^l_{m,n}(R) \diff{R} &= \frac{1}{8\pi^2} \int_0^{2\pi} \int_0^\pi \int_0^{2\pi} e^{-im\alpha} d^l_{m,n}(\beta) e^{-in\gamma} \sin\beta \diff{\alpha} \diff{\beta} \diff{\gamma} \\
		&= \frac{1}{2} \delta_m^0 \delta_n^0 \int_0^\pi d^l_{0,0}(\beta) \sin\beta \diff{\beta} = \frac{1}{2} \delta_m^0 \delta_n^0 \int_0^\pi P_l(\cos\beta) \sin\beta \diff{\beta} = \delta_m^0 \delta_n^0 \delta_l^0.
	\end{align*}
	Thus, \eqref{eqn:F_J calculation} reduces to \eqref{eqn:F_J}.
\end{proof}

Let $p(\Omega,t) = \int_{R\in\SO}p(R,\Omega,t)\diff{R}$ be the marginal density of $\Omega$, and denote $F_{i,j,k}[p](t)$ as its Fourier coefficient.
Then using Lemma \ref{lemma:marginal Fourier}, $F_{i,j,k}[p](t)$ satisfied the ordinary differential equation
\begin{align} \label{eqn:FP marginal Fourier}
	\frac{\diff{F_{i,j,k}[p]}}{\diff{t}} =  \frac{\diff{F^0_{0,0,i,j,k}[p]}}{\diff{t}}.
\end{align}
Now let us calculate $\frac{\diff{F_{i,j,k}[p]}}{\diff{t}}$ by substituting \eqref{eqn:FP Fourier} into \eqref{eqn:FP marginal Fourier}.
First, the first term on the right hand side of \eqref{eqn:FP Fourier} is
\begin{align} \label{eqn:dF marginal}
	\frac{\diff{F^0_{0,0,i,j,k}[p]}}{\diff{t}} &= -\sum_{\alpha=1}^3 F^0_{0,0,i,j,k}[\liediff_\alpha(\Omega_\alpha p)] + \sum_{\alpha=1}^3 F^0_{0,0,i,j,k}\left[ \frac{\partial}{\partial \Omega_\alpha}[(\Omega\times J\Omega + mg\rho\times R^Te_3)_\alpha p] \right] \nonumber \\
	&\qquad\qquad +\sum_{\alpha,\beta=1}^3 G_{\alpha\beta} F^0_{0,0,i,j,k}\left[ \frac{\partial^2 p}{\partial\Omega_\alpha \partial\Omega_\beta} \right] \nonumber \\
	&= -\sum_{\alpha=1}^3 F_{i,j,k} \left[ \int_{R\in\SO} \liediff_\alpha(\Omega_\alpha p(R,\Omega,t)) \diff{R} \right] + \sum_{\alpha=1}^3 F_{i,j,k} \left[ \int_{R\in\SO} \frac{\partial}{\partial \Omega_\alpha}[(\Omega\times J\Omega)_\alpha p(R,\Omega,t)] \diff{R} \right] \nonumber \\
	&\qquad\qquad + \sum_{\alpha=1}^3 F^0_{0,0,i,j,k}\left[ \frac{\partial}{\partial \Omega_\alpha}[(mg\rho\times R^Te_3)_\alpha p] \right] + \sum_{\alpha,\beta=1}^3 G_{\alpha\beta} F_{i,j,k} \left[ \int_{R\in\SO} \frac{\partial^2 p(R,\Omega,t)}{\partial\Omega_\alpha \partial \Omega_\beta} \diff{R} \right] \nonumber \\
	&= -\sum_{\alpha=1}^3 F_{i,j,k} \left[ \frac{\partial}{\partial \Omega_\alpha} [(\Omega\times J\Omega)_\alpha p(\Omega,t)] \right] + \sum_{\alpha=1}^3 F^0_{0,0,i,j,k}\left[ \frac{\partial}{\partial \Omega_\alpha}[(mg\rho\times R^Te_3)_\alpha p(R,\Omega,t)] \right] \nonumber \\
	&\qquad\qquad + \sum_{\alpha,\beta=1}^3 G_{\alpha\beta} F_{i,j,k} \left[ \frac{\partial^2 p(\Omega,t)}{\partial \Omega_\alpha \partial \Omega_\beta} \right].
\end{align}
where the last equality comes the from the following lemma
\begin{lemma}
	Let $f\in C^1(\SO)$, then $\int_{R\in\SO} \liediff_i f \diff{R} = 0$ for $i=1,2,3$.
\end{lemma}
\begin{proof}
	It suffices to check the following:
	\begin{align*}
		\int_{R\in\SO} \liediff_i f\diff{R} &= \int_{R\in\SO} \frac{\diff{}}{\diff{t}} \bigg\lvert_{t=0} f(R\exp(t\hat{e}_i)) \diff{R} \\
		&= \lim\limits_{t\to 0} \int_{R\in\SO} f(R\exp(t\hat{e}_i)) \diff{R} - \int_{R\in\SO} f(R) \diff{R} = 0.
	\end{align*}
\end{proof}

Unfortunately, the second term on the right hand side of \eqref{eqn:dF marginal} cannot be written in terms of $F_{i,j,k}[p(\Omega,t)]$.
But it can be simplified using \eqref{eqn:F_J calculation} without resorting to the product rule of Fourier transform on $\SO$.
By Lemma \ref{lemma:marginal Fourier} and \eqref{eqn:F_J calculation}, we have
\begin{align*}
	F^0_{0,0,i,j,k}[R_{ab}p(R,\Omega,t)] &= F_{i,j,k}\left[ \int_{R\in\SO} R_{ab}p(R,\Omega,t) \diff{R} \right] \\
	&= \sum_{l=0}^\infty \sum_{m,n=-l}^l (2l+1) F^l_{m,n,i,j,k}[p(R,\Omega,t)] \int_{R\in\SO} R_{ab}D^l_{m,n}(R) \diff{R}.
\end{align*}
Using Euler angles (body-fixed 3-2-3), $R$ can be written as
\begin{align}
	R(\alpha,\beta,\gamma) = \begin{bmatrix}
		\cos\alpha\cos\beta\cos\gamma-\sin\alpha\sin\gamma & -\cos\alpha\cos\beta\sin\gamma-\sin\alpha\cos\gamma & \cos\alpha\sin\beta \\
		\sin\alpha\cos\beta\cos\gamma+\cos\alpha\sin\gamma & -\sin\alpha\cos\beta\sin\gamma+\cos\alpha\cos\gamma & \sin\alpha\sin\beta \\
		-\sin\beta\cos\gamma & \sin\beta\sin\gamma & \cos\beta
	\end{bmatrix}.
\end{align}
Note that $\int_0^{2\pi}\sin\alpha e^{-im\alpha} \diff{\alpha} = 2\pi\left( \frac{i}{2}\delta_m^{-1} - \frac{i}{2}\delta_m^1 \right)$, and $\int_0^{2\pi}\cos\alpha e^{-im\alpha} \diff{\alpha} = 2\pi\left( \frac{1}{2}\delta_m^{-1} + \frac{1}{2}\delta_m^1 \right)$.
So we have the following explicit calculations
\begin{align*}
	\int_{R\in\SO} R_{11}D^l_{m,n}(R) \diff{R} &= \frac{1}{4} \int_0^\pi \left( \left(d^l_{1,1}(\beta) + d^l_{1,-1}(\beta)\right)\cos\beta + d^l_{1,1}(\beta) - d^l_{1,-1}(\beta) \right) \sin\beta \diff{\beta}  \\
	\int_{R\in\SO} R_{12}D^l_{m,n}(R) \diff{R} &= 0 \\
	\int_{R\in\SO} R_{13}D^l_{m,n}(R) \diff{R} &= 0 \\
	\int_{R\in\SO} R_{21}D^l_{m,n}(R) \diff{R} &= 0 \\
	\int_{R\in\SO} R_{22}D^l_{m,n}(R) \diff{R} &= \frac{1}{4} \int_0^\pi \left( \left(d^l_{1,1}(\beta) - d^l_{1,-1}(\beta)\right)\cos\beta + d^l_{1,1}(\beta) + d^l_{1,-1}(\beta) \right) \sin\beta \diff{\beta} \\
	\int_{R\in\SO} R_{23}D^l_{m,n}(R) \diff{R} &= 0 \\
	\int_{R\in\SO} R_{31}D^l_{m,n}(R) \diff{R} &= 0 \\
	\int_{R\in\SO} R_{32}D^l_{m,n}(R) \diff{R} &= 0 \\
	\int_{R\in\SO} R_{33}D^l_{m,n}(R) \diff{R} &= \frac{1}{2} \int_0^\pi P_l(\cos\beta) \cos\beta \sin\beta \diff{R}.
\end{align*}

\section{Conditional Density}

Let $p(R,t\lvert\Omega)$ be the conditional density and $F^l_{m,n}[p](\Omega,t)$ be its Fourier coefficient.
For any fixed $\Omega\in\real^3$, the first dynamical equation can be solved analytically as
\begin{align*}
	R(t+\Delta t) = R(t)\exp(\hat{\Omega}\Delta t) \triangleq R(t)\Delta R(\Omega).
\end{align*}
So we have the following density propagation rule
\begin{align*}
	p(R,t+\Delta t\lvert\Omega) = p(R\Delta R(\Omega)^T, t\lvert\Omega) \triangleq \tilde{p}(R,t\lvert\Omega).
\end{align*}
By lemma \ref{lemma:fixed rotation}, we have $F^l[p](\Omega,t+\Delta t) = F^l[p'](\Omega,t) = F^l[p](\Omega,t)\bar{D}^l(\Delta R(\Omega))$, which is the propagation rule for the Fourier coefficients of the marginal density.
\begin{lemma} \label{lemma:fixed rotation}
	Let $R'$ be a fixed rotation, then $F^l[f(RR'^T)] = F^l[f(R)]\bar{D}^l(R')$.
\end{lemma}
\begin{proof}
	The Fourier coefficient of $f(RR'^T)$ can be calculated directly as
	\begin{align*}
		F^l[f(RR'^T)] &= \int_{R\in\SO} f(RR'^T) \bar{D}^l(R) \diff{R} = \int_{R\in\SO} f(R) \bar{D}^l(RR') \diff{R} \\
		&= \int_{R\in\SO} f(R) \bar{D}^l(R) \diff{R} \bar{D}^l(R') = F^l[f(R)]\bar{D}^l(R').
	\end{align*}
\end{proof}

\end{document}

