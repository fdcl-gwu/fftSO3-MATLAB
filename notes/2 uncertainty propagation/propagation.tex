\documentclass[12pt]{article}

\usepackage{amssymb,amsmath,amsthm}
\usepackage{bm}
\usepackage{graphicx,subcaption}
\usepackage[letterpaper, top=1in, left=1in, right=1in, bottom=1in]{geometry}

\newtheorem{definition}{Definition}
\newtheorem{theorem}{Theorem}
\newtheorem{lemma}{Lemma}
\newtheorem{remark}{Remark}

\newcommand{\SO}{\ensuremath{\mathrm{SO}(3)}}
\newcommand{\tr}[1]{\ensuremath{\mathrm{tr}\left( #1 \right)}}
\newcommand{\abs}[1]{\ensuremath{\left| #1 \right|}}
\newcommand{\diff}[1]{\mathrm{d}#1}
\newcommand{\vect}[1]{\ensuremath{\mathrm{vec}\left[ #1 \right]}}

\newcommand{\liediff}{\mathfrak{d}}
\newcommand{\dft}{\mathcal{F}}
\newcommand{\real}{\ensuremath{\mathbb{R}}}

\title{\vspace{-4ex}\textbf{Uncertainty Propagation on SO(3)\vspace{-4ex}}}
\date{}

\begin{document}

\maketitle

Consider the stochastic differential equation
\begin{align*}
	R^T\diff{R} = \hat{\Omega}(R,t)\diff{t} + (H(t)\diff{W_t})^\wedge.
\end{align*}
Let $p(R,t)$ be the density function of $R$ at time $t$, then $p$ satisfies the Fokker-Planck equation
\begin{align}
	\frac{\partial p(R,t)}{\partial t} = -\sum_{i=1}^3 \liediff_i(\Omega_i(R,t)p(R,t)) + \frac{1}{2} \sum_{i,j=1}^3 \liediff_i\liediff_jp(R,t) G_{ij}(t), \label{eqn:FP}
\end{align}
where $G = H(t)H(t)^T$.
Let $\bar{p}_{m,n}^l(t)$ be the Fourier coefficient of $p(R,t)$ at time $t$, then we have the approximation
\begin{align*}
	p(R,t) = \sum_{l=0}^{B-1} \sum_{m,n=-l}^l \bar{p}_{m,n}^l(t) D_{m,n}^l(R).
\end{align*}
Using the differentiation rule, \eqref{eqn:FP} can be approximated by
\begin{align*}
	\frac{\diff{}}{\diff{t}} \bar{p}^l(t) = -\sum_{i=1}^3 \dft^l[\Omega_i(R,t)p(R,t)] (u_i^l)^T + \frac{1}{2}\sum_{i,j=1}^3 G_{ij} \bar{p}^l(t) (u_j^l)^T(u_i^l)^T.
\end{align*}

\section{Drift}

First look at $\mathcal{F}^l[\Omega_i(R,t)p(R,t)]$.
Based on the product rule, we have
\begin{align}
	\mathcal{F}^l(\Omega_i(R,t)p(R,t)) = \sum_{\substack{\abs{l_1-l_2} \leq l \leq l_1+l_2 \\ 0 \leq l_1 \leq B-1 \\ 0 \leq l_2 \leq B-1}} \frac{(2l_1+1)(2l_2+1)}{2l+1} C_{l_1,l_2,l}^T \bar{\Omega}_i^{l_1} \otimes \bar{p}^{l_2} C_{l_1,l_2,l}, \label{eqn:prod}
\end{align}
where $C_{l_1,l_2,l}\in\real^{(2l_1+1)(2l_2+1) \times (2l+1)}$ is defined in the previous note.
Split the rows of $C_{l_1,l_2,l}$ equally into $2l_1+1$ pieces, and denote $C_{l_1,l_2,j,l} \in \real^{(2l_2+1) \times (2l+1)}$ as the $(j+l_1+1)$-th piece for $j=-l_1,\ldots,l_1$.
Then \eqref{eqn:prod} can be written as
\begin{align} \label{eqn:l,l2}
	\mathcal{F}^l(\Omega_i(R,t)p(R,t)) = \sum_{\substack{\abs{l_1-l_2} \leq l \leq l_1+l_2 \\ 0 \leq l_1 \leq B-1 \\ 0 \leq l_2 \leq B-1}} \frac{(2l_1+1)(2l_2+1)}{2l+1} \sum_{j,k=-l_1}^{l_1} \bar{\Omega}_{j,k,i}^{l_1} C_{l_1,l_2,j,l}^T \bar{p}^{l_2} C_{l_1,l_2,k,l}.
\end{align}

Element-wise, it becomes
\begin{align*}
	\mathcal{F}_{m,n}^l(\liediff_i(\Omega_ip)) &= \sum_{\substack{\abs{l_1-l_2} \leq l \leq l_1+l_2 \\ 0 \leq l_1 \leq B-1 \\ 0 \leq l_2 \leq B-1}} \frac{(2l_1+1)(2l_2+1)}{2l+1} \sum_{j,k=-l_1}^{l_1} \bar{\Omega}_{j,k,i}^{l_1} \sum_{p,q=-l_2}^{l_2} (C_{l_1,l_2,j,l})_{p,m}  (C_{l_1,l_2,k,l}(u_i^l)^T)_{q,n} \bar{p}_{p,q}^{l_2} \\
	&= \sum_{l_2=0}^{B-1} \sum_{p,q=-l_2}^{l_2} \bar{p}_{p,q}^{l_2} \sum_{\substack{\abs{l_1-l_2} \leq l \leq l_1+l_2 \\ 0 \leq l_1 \leq B-1}} \sum_{j,k=-l_1}^{l_1} \frac{(2l_1+1)(2l_2+1)}{2l+1} \bar{\Omega}_{j,k,i}^{l_1} (C_{l_1,l_2,j,l})_{p,m} (C_{l_1,l_2,k,l}(u_i^l)^T)_{q,n}.
\end{align*}
Denote $(C_{l_1,l_2,l})_{p,m} \in \real^{2l_1+1}$ as the $(p+l_2+1+(\alpha-1)(2l_2+1))$-th row for $\alpha = 1,\ldots,2l_1+1$, and $(m+l+1)$-th column of $C_{l_1,l_2,l}$.
Similarly denote $(C_{l_1,l_2,l}(u_i^l)^T)_{q,n}$ as the $(q+l_2+1+(\alpha-1)(2l_2+1))$-th row for $\alpha = 1,\ldots,2l_1+1$, and $(n+l+1)$-th column of $C_{l_1,l_2,l}(u_i^l)^T$.
Then the above element-wise formula can be written as
\begin{align*}
	\mathcal{F}_{m,n}^l(\liediff_i(\Omega_ip)) = \sum_{l_2=0}^{B-1} \sum_{p,q=-l_2}^{l_2} \bar{p}_{p,q}^{l_2} \sum_{\substack{\abs{l_1-l_2} \leq l \leq l_1+l_2 \\ 0 \leq l_1 \leq B-1}} \frac{(2l_1+1)(2l_2+1)}{2l+1} \tr{\bar{\Omega}_i^{l_1} (C_{l_1,l_2,l}(u_i^l)^T)_{q,n} (C_{l_1,l_2,l})_{p,m}^T}
\end{align*}

From \eqref{eqn:l,l2}, we can also derive the following matrix expression
\begin{align*}
	&\vect{\mathcal{F}^l(\liediff_i(\Omega_ip))} = \sum_{l_2=0}^{B-1} \sum_{\substack{\abs{l_1-l_2} \leq l \leq l_1+l_2 \\ 0 \leq l_1 \leq B-1}} \frac{(2l_1+1)(2l_2+1)}{2l+1} \sum_{j,k=-l_1}^{l_1} \bar{\Omega}_{j,k,i}^{l_1} \vect{C_{l_1,l_2,j,l}^T \bar{p}^{l_1} C_{l_1,l_2,k,l}(u_i^l)^T} \\
	&\qquad = \sum_{l_2=0}^{B-1} \sum_{\substack{\abs{l_1-l_2} \leq l \leq l_1+l_2 \\ 0 \leq l_1 \leq B-1}} \frac{(2l_1+1)(2l_2+1)}{2l+1} \sum_{j,k=-l_1}^{l_1} \bar{\Omega}_{j,k,i}^{l_1} (u_i^l C_{l_1,l_2,k,l}^T \otimes C_{l_1,l_2,j,l}^T) \vect{\bar{p}^{l_2}} \\
	&\qquad = \sum_{l_2=0}^{B-1} \sum_{\substack{\abs{l_1-l_2} \leq l \leq l_1+l_2 \\ 0 \leq l_1 \leq B-1}} \frac{(2l_1+1)(2l_2+1)}{2l+1} \sum_{j,k=-l_1}^{l_1} \bar{\Omega}_{j,k,i}^{l_1} (u_i^l \otimes I^l)(C_{l_1,l_2,k,l}^T \otimes C_{l_1,l_2,j,l}^T) \vect{\bar{p}^{l_2}} \\
	&\qquad = (u_i^l \otimes I^l) \sum_{l_2=0}^{B-1} \left( \sum_{\substack{\abs{l_1-l_2} \leq l \leq l_1+l_2 \\ 0 \leq l_1 \leq B-1}} \frac{(2l_1+1)(2l_2+1)}{2l+1} \sum_{j,k=-l_1}^{l_1} \bar{\Omega}_{j,k,i}^{l_1} (C_{l_1,l_2,k,l}^T \otimes C_{l_1,l_2,j,l}^T) \right) \vect{\bar{p}^{l_2}},
\end{align*}
where $I^l = I_{(2l+1) \times (2l+1)}$.

\section{Diffusion}

\begin{align*}
	\vect{\bar{p}^l(u_j^l)^T(u_i^l)^T} = \left( (u_i^lu_j^l) \otimes I^l \right) \vect{\bar{p}^l}
\end{align*}

\end{document}

